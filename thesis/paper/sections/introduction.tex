% !TEX root = ../thesis.tex

\chapter{Introduction}

The leap in IoT (Internet of Things) in the recent decade has opened doors to a new era in information technology. New realms of interconnected devices and device families introduced different ways of interacting with information. Emerging new tools, techniques, frameworks, libraries, and sdk's made it possible to build consumer-facing products in ways that were never possible before. These advancements attracted many users and developers, and helped create a large and diverse market of consumer products, goods and services. However, it did not come without some hurdles. In this thesis, we group these hurdles under three main categories; consuming data, privacy \& security, and providing data. Consuming data section discusses the problems faced by end-users due to the inconsistencies of user interface and experience, and lack of device and accessibility support. The Privacy and Security section argues why there are no truly private and secure environments. Providing data section gives some insights on the challenges faced by data and service providers in creating front-end applications.

\paragraph{Consuming data}

The simplicity of consuming data is lost within the complexity of the modern day applications; nowadays something as simple as checking the weather, reading a news article, browsing an image gallery, buying a product or service, filling out a form etc. could be a frustrating and time consuming task. One of the common reasons for this is the inconsistencies and errors in front-end implementations. In the open market, where everyone can create user-facing applications, there exists little to no enforcement or quality control measures on how a user-interface should look like, or how it should behave. The way the same functions are presented can greatly differ based on the implementations. For instance, a navigational menu component in a website could be on the top while in another could have it on the left or right, some could offer a hamburger-menu style functionality where the user has to click / tap on the icon to toggle it on and off, in some swiping from left of the screen gesture might toggle the navigation menu on, while some other might implement the same gesture as “go back” function, and so on. In some cases the developer can choose a bad selection of a color palette causing some components to blend into the background or make it hard for the users to see. Users oftentimes have to take a second to adjust to every different experience from every different provider. While some providers with higher development budgets create flawless experiences for their users, this is not always the case and there is no guarantee.

Another major hurdle in consuming data for users is device and screen size support. Providers typically need to spend significant extra effort to support different devices and screen sizes, should they want such support in the first place. Many providers choose not to have this support in order to reduce development cost, leaving some users with bad experiences or no experience at all. Different levels of support for different environments might cause confusion and frustration, for instance some providers might offer a native mobile application with native mobile gestures, while some might have a PWA (Progressive Web App) that comes with native-like gestures and some might offer a mobile-web experience that may or may not come with smooth gestures at all. Another major experience difference between mobile applications and web-based applications is the style of navigation. Users who are accustomed to using native mobile applications could expect the navigational history to be retained between tabs, and get frustrated when they realize that this is not the case on a mobile-web application. At times improperly handled navigation in a SPA (Single Page Application) might even cause the native “back” functionality to throw the user out of the application back to the previous one.

Creating accessible front-ends is another thing that providers have to spend significant efforts on, so much so that there are even developers who specialize only in this field. Creating accessible applications is not always the first priority on many development projects due to the costs and efforts involved. The diversity in user interfaces has a direct effect on the accessibility aspect, as it is another implementation detail that is left to the developer. There are many different ways of creating accessible user interfaces; different implementations for different kinds of accessibility needs, and different ways of implementing each one of them. For instance in a web application, adding “focus” states to DOM elements is the most basic form of accessibility implementation that allows users to tab into a specific element to be able to interact with it. Designing the order in which they receive focus is a whole other dimension. 

\paragraph{Privacy and Security}

As mentioned before, there is a lack of a systematic enforcement mechanism for front-end applications. There are some protective measures in place, however it is most likely the case that these measures do not guarantee a safe, secure and private environment. For instance, most popular application stores typically have policies and guidelines on what the applications they distribute are allowed to do, but offending practices, especially the non-obvious ones, likely flies under the radar as most app stores do not require the source codes to be provided. Browsers and operating systems block suspicious activities to some extent, but they do not (or cannot) interfere with the legitimate (or legitimate-looking) ones. Governments implement various forms of cybersecurity laws to protect users, but of course laws are for the law-abiding, and as long as these laws cannot be enforced effectively, there is no safe environment for the users. The bottom line is, as long as users are required to execute code on their devices, they cannot be truly safe.

\paragraph{Providing Data}

The word Providers refers to anyone who offers data and services to end users, and it is safe to say that the problems discussed under the Consuming Data section are shared also by the providers. The complexity of building a decent, well designed, accessible front-end experience, not even once but once per every client built for different platforms forces providers to choose between supporting multiple devices, following best practices, creating a good experience etc. It is typically the case that at the start providers cannot have it all, as it requires significant costs and large teams. 

Code sharing is one of the recent trends in front-end development that has been gaining some traction, which aims to solve some of these aforementioned problems; by using technologies such as react-native, react-native-web and Flutter, one can have a single codebase that produces applications for multiple environments. Modern front-end libraries / frameworks typically abstracts away the view layer, making it possible to create applications for different environments by swapping the view layer out and wiring it up to the logic layer, and this seems to be the current status quo in cross-platform application development. The issue that comes with this approach however is the difficulty in reducing the codebases for different environments into one, as it is hard to address different requirements and features of each platform. Even though it is possible to share some extent of the code, at times platforms are conceptually different and the software for these platforms needs to be built in a different way. 

Another pain point faced by providers is maintenance. The world of front-end development is an extremely fast growing and evolving ecosystem. The changes are very frequent, and at times breaking. There are thousands of tools, libraries, frameworks, SDK’s available at the fingertips of developers at no cost. It is a common practice to make use of these libraries as they help with the development significantly. However, the diversity in the libraries used to build the software likely results in a diversity of maintenance problems. Even the most well-tested and maintained libraries could break after an update, causing headache for the developers and hardship for the users.