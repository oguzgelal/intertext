% !TEX root = ../thesis.tex

\chapter{Intertext}

Intertext is a platform based on a straightforward premise; it is a family of front-end applications that can interpret IUIDL and generate appropriate front-ends for the host platform. Simply put, a provider wanting to create an Intertext application creates a generic backend that generates IUIDL, and serves it from an endpoint, say at https://intertext.example.com. Users who wish to use this application pulls up an Intertext client on their preferred device or environment and visit this domain just like in a web browser. The Intertext client then requests this domain, fetch the IUIDL served by this endpoint, and generate the user interface as per the instructions received. Moreover, rather than rendering a simple static view, it performs some tasks such as accepting user input, navigating to different screens, making additional requests to fetch more data, keeping the UI updated and reading and writing some data to users local storage; all of which is again orchestrated based on the instructions received by the backend in IUIDL syntax. Intertext has multiple software clients built natively for various platforms that can interpret IUIDL most appropriately to the host device or platform. For instance, users browsing an Intertext app through a smartphone receives an experience optimized for touch screens, command-line interface client users receive an optimized experience for a text-based interface or a user browsing from a low-end device with limited capabilities use the version optimized for low-performance devices to get a comfortable viewing experience and so on. 

Intertext gives providers a set of components to build and serve their UIs for their services.  Given the involvement of a backend to handle all the logic, this service can be anything. In other words, users can enjoy their todo lists, habit tracker, notes, calendar, email client, social apps, news, weather and so on, all through one single app, using the Intertext client of their choice. IUIDL is agnostic of styles; for instance, a provider could create a "Button" component and specify what it says and what it does, but they cannot specify its looks, thus creating unified styles and user experience. This unity brings many advantages, most notably, consistency. Every single application on Intertext looks and feels the same, made out of components that users are familiar with. Customizability is another significant advantage; users can adjust the look and feel of these components, allowing them to personalize their browsing experience for all applications on the platform. Lastly, all components are accessible from the ground up. Thanks to this approach, the accessibility implementations are not the developers' responsibility; therefore, users with specific needs are guaranteed to have the accessibility features they need for every application Intertext platform. 

Intertext is an open platform, and IUIDL is well documented for developers to build client applications, allowing the community to develop specific client applications that can interpret IUIDL in meaningful ways to respond to specific needs. Whether it is a VUI (Voice User Interface) client, a Tangible UI client, or even clients built for particular devices with specific requirements for targeting various communities or use cases, they can all support all existing Intertext applications served from backend services.  

When it comes to privacy and security, the bottom line is that Intertext takes away the providers' ability to execute code on users devices. Applications running on Intertext clients are expected to implement all their application logic on the server-side, serve some instructions to the Intertext clients on what to do, how to function, what to show the user and so on. These instructions are a part of the IUIDL, and they are purely XML-based data; no executables are allowed from the providers. Nevertheless, Intertext allows applications to use a specific set of functions required to build a meaningful front-end application. For instance, an application can instruct an Intertext client to take actions like storing data, reading previously stored data, or making requests. Applications are also allowed to ask for permissions for things like accessing the camera (for devices that have one), notifications or location. 

Intertext by default blocks cross-origin requests, meaning that a provider serving data from an origin can only ask the Intertext client to make a request to the same origin. Storage access is also bound to the origin; providers serving data from an origin cannot access the data stored by another origin, preventing users from being tracked across the web for targeted advertisements. Furthermore, all requests that go back and forth are displayed to the end-users in a user-friendly way for transparency. This controlled approach guarantees the maximum level of privacy and security.

As convenient as Intertext is for end-users, this project aims to create advantages as attractive for data or service providers. It may not be the best option for all cases, such as front-end-heavy applications requiring client-side computations, custom styles or advanced graphics. However, for most cases, it has many advantages over building and maintaining front-end applications from scratch. Creating front-end business logic in the backend might be an overhead; however, the effort required is nothing compared to all the hurdles mentioned earlier. Intertext is agnostic of where the data is coming from and has no opinions on how it is generated. Therefore, the providers can easily use their existing backend services to add in the front-end logic. IUIDL is XML-based, and working with data is an essential part of every application; it is a light effort but greatly rewarding. Once a provider starts serving their front-end in IUIDL, that means they immediately get front-end applications for every platform that Intertext supports. Furthermore, as more Intertext clients get built, either officially or by the community, their applications immediately start working with those clients without requiring any change. 

Intertext creates equal opportunities for everyone. The advancements in backend technologies these days made creating backend services possible in ways that were never possible before. Service providers such as AWS, Google Cloud and Azure made spinning up a backend infrastructure with necessary components as simple as a few clicks, all without requiring DevOps skills. Their generous free tiers, scalable infrastructures and the recently emerging serverless technology pulled the costs down so significantly that anyone familiar with backend technologies can create an application that could scale up to serve hundreds of thousands of users. However, one of the biggest blockers in launching applications that could gain popularity and evolve into a successful startup is arguably the branding and quality of the front-end applications. An application that suffers from user-facing problems mentioned earlier does not leave a good impression and creates a bad image, and it is relatively uncommon for such applications to be taken seriously, regardless of the quality of the data and services provided. Those who can achieve an exceptional user-facing presence while targeting multiple devices and platforms are commonly large companies with high development budgets. These costs hurt indie developers, particularly backend developers who do not specialize in front-end development but can create applications as personal projects that could easily qualify as a good product. This project aims to remove this hegemony of "judging a book by the cover"; when all applications look the same and feel the same way, they are judged the same way. Not only indie developers have better chances of success, but there are also be more options for everyone.