% !TEX root = ../../thesis.tex

\section{UIDLs} \label{relatedUIDLs}

A UIDL can be defined as having two separate parts: a syntax that describes the user interface characteristics, and semantics that defines what these characteristics mean. They often share the common goal of describing a UI without targeting any particular programming language or platform; nevertheless, the end-goal of UIDLs often varies \cite{XMLCompliantUIDLs}. UIDLs often uses XML or a similar markup language/notation that later gets transpiled into a programming language or is processed by a software to automatically or semi-automatically be translated into a UI, a visualisation, or any by-product depending on the goal of the project. A UIDL can be thought of as a tool designed to achieve a particular goal or to achieve a common goal in a particular way. 

Souchon, Nathalie and Vanderdonckt et al., 2003 \cite{XMLCompliantUIDLs}, and later Guerrero-Garcia, Josefina and Gonzalez-Calleros et al., 2009 \cite{UIDLTheoreticalSurvey} argued \textit{"there is a plethora of UIDLs that are widely used, with different goals and different strengths"}. There are many ways of classifying and categorising existing UIDLs, as can be seen in \textit{A review of XML-compliant user interface description languages, 2003} \cite{XMLCompliantUIDLs}, and \textit{A theoretical survey of user interface description languages: Preliminary results, 2009} \cite{UIDLTheoreticalSurvey}. However, in this section, to stay relevant to the comparison to Intertext, we group UIDLs under three main categories: compiled, compiled with runtime support, and interpreted UIDLs. Most of them belong to the first category, allowing us to focus on the general picture instead of on a case-by-case basis. This separation narrows down the comparison to a select few UIDLs, which we investigate in more detail.

\subsubsection{Compiled UIDLs}

\subsubsection{Compiled UIDLs with Runtime Support}

\subsubsection{Interpreted UIDLs}
