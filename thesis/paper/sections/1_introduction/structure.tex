% !TEX root = ../../thesis.tex

\section{Structure} \label{structure}

\begin{enumerate}
    
    \item \textbf{\nameref{relatedWork}}: In this section, we first explore similar research that has been done. We evaluate several of the UIDLs, discuss the similarities and what separates Intertext from them. We also introduce other libraries, frameworks, tools, services and other solutions similar to Intertext.

    \item \textbf{\nameref{solution}}: In this section, we thoroughly discuss the details of the Intertext project. We first walk through the design principles and explain the purpose behind each, and what problems in the \nameref{problemStatement} are they aiming to address. Then, we introduce Intertext User Interface Description Language (IUIDL), our XML-based markup language. Finally, we talk about Intertext clients, what they are, and how they work.

    \item \textbf{\nameref{implementation}}: This section talks about the overall architecture of Intertext. We first talk about Intertext engine and several other core components of Intertext. Then, we list the Components and Commands that IUIDL offers out of the box.
    
    \item \textbf{\nameref{evaluation}}: In this section we give example use-cases scenarios for the users, and walk through the sample RecipeApp that intends to serve as an example use-case for developers in technical sense. Then, we move on the user evaluation; first we explain our methodology, and then we share the results.

    \item \textbf{\nameref{discussion}}: In this section, we explain how we addressed the problems in \nameref{problemStatement} and how we answer the questions in \nameref{researchQuestions}. Then, we talk about the limitations.

    \item \textbf{\nameref{conclusion}}: In this section we list our main contributions with Intertext project, and give a non-exclusive list of our future plans.
    
\end{enumerate}