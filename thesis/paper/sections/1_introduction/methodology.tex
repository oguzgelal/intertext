% !TEX root = ../../thesis.tex

\section{Methodology} \label{methodology}

To identify and effectively address all these problems, we utilized the Design Science Research Methodology (DSRM) \cite{DSRM} to carry out our research. DSRM is a commonly adopted research framework that provides six executable steps to guide the researcher in being consistent with prior work and provides a theoretical process for doing the research, and guides through presenting and evaluating the outcome. 

The first step of DSRM is \textbf{identifying the problem and the motivation}. It is not uncommon for research and inventions to stem from a problem or a necessity. Similarly, in our case, the problem was out there; we only needed to focus on it to better understand the issues we are tackling. In \nameref{problemStatement}, we discussed and justified what problems we are attempting to solve with Intertext. Then we proceeded to the next step, \textbf{defining the objectives for a solution}. As we narrowed down the problems that we attempt to solve in \nameref{problemStatement}, we drafted our objectives to approach this problem. We explained these objectives in detail in \nameref{researchQuestions}. Then, we were ready to take the next step of \textbf{design and development}. In the sections \nameref{solution} and \nameref{implementation}, we explained in detail our development process, all the challenges and design decisions that we have taken in order to address the problems in the best way possible. 

Once we had a working prototype, we created a simple Intertext application, RecipeApp, as explained in detail in \nameref{evaluation} section. This application served as a demonstration for developers on how IUIDL could be dynamically served from a backend. We also introduced Intertext to end-users, communicated its goals and motivations, and then collected feedback. These concludes the final three steps of \textbf{demonstration}, \textbf{evaluation} and \textbf{communication} into one.